\begin{filecontents*}{example.eps}
%!PS-Adobe-3.0 EPSF-3.0
%%BoundingBox: 19 19 221 221
%%CreationDate: Mon Sep 29 1997
%%Creator: programmed by hand (JK)
%%EndComments
gsave
newpath
  20 20 moveto
  20 220 lineto
  220 220 lineto
  220 20 lineto
closepath
2 setlinewidth
gsave
  .4 setgray fill
grestore
stroke
grestore
\end{filecontents*}
%
\RequirePackage{fix-cm}
%
%\documentclass{svjour3}                     % onecolumn (standard format)
%\documentclass[smallcondensed]{svjour3}     % onecolumn (ditto)
\documentclass[smallextended]{svjour3}       % onecolumn (second format)
%\documentclass[twocolumn]{svjour3}          % twocolumn
%
\smartqed  % flush right qed marks, e.g. at end of proof
%


%
% \usepackage{mathptmx}      % use Times fonts if available on your TeX system
%

\usepackage{graphicx}
\usepackage[english]{babel}
\usepackage{blindtext}
\usepackage{algorithm}
\usepackage{algpseudocode}
\usepackage{bm}
\usepackage{multirow}
\usepackage{amsmath}
\usepackage{mathtools}
\usepackage{mathrsfs}
\usepackage{theorem}
\usepackage{amsfonts,amssymb}
\usepackage{multicol}


\usepackage{extarrows}

\usepackage[inline]{enumitem}



\bibliographystyle{splncs03}
%





\newcommand\revision{\textcolor{red}}



\newcommand\arrangespace{\vspace{+1em}}
\newcommand\reducespace{\vspace{-1.35em}}
\newcommand\reducelittlespace{\vspace{-0.7em}}
\newcommand\codeindent{\ \ \ }



%
%\makeatletter
%\newcommand\footnoteref[1]{\protected@xdef\@thefnmark{\ref{#1}}\@footnotemark}
%\makeatother



\newcommand\calE{\ensuremath{\mathcal{E}}}
\newcommand\calM{\ensuremath{\mathcal{M}}}
\newcommand\calK{\ensuremath{\mathcal{K}}}
\newcommand\calP{\ensuremath{\mathcal{P}}}
\newcommand\calR{\ensuremath{\mathcal{R}}}
\newcommand\calI{\ensuremath{\mathcal{I}}}
\newcommand\calS{\ensuremath{\mathcal{S}}}





\newcommand\e{\textsf{E}}
\newcommand\kac{\textsf{KAC}}


\newcommand\kaf{\textsf{KAF}}
\newcommand\kafsp{\textsf{KAFSP}}
\newcommand\kafsf{\textsf{KAFSF}}
\newcommand\kafv{\textsf{KAFv}}
\newcommand\kafvsp{\textsf{KAFvSP}}
\newcommand\kafvsf{\textsf{KAFvSF}}
\newcommand\kafw{\textsf{KAFw}}
\newcommand\kafwsp{\textsf{KAFwSP}}
\newcommand\kafwsf{\textsf{KAFwSF}}
\newcommand\cmt{\textsf{CMT}\xspace}
\newcommand\D{\Delta}
\newcommand\Zn{\{0,1\}^n}
\newcommand\xor{\oplus}
\newcommand\mul{\otimes}
\newcommand\ga{\gamma}
\newcommand\wf{w}    % KDF for the whitening keys
%\newcommand\wf{\psi}    % KDF for the whitening keys



\newcommand\bone{\text{(B-1)}\xspace}
\newcommand\btwo{\text{(B-2)}\xspace}
\newcommand\bthree{\text{(B-3)}\xspace}
\newcommand\bthreeone{\text{(B-31)}\xspace}
\newcommand\bthreetwo{\text{(B-32)}\xspace}
\newcommand\bthreethree{\text{(B-33)}\xspace}
\newcommand\bfour{\text{(B-4)}\xspace}
\newcommand\bfive{\text{(B-5)}\xspace}
\newcommand\bsix{\text{(B-6)}\xspace}
\newcommand\bseven{\text{(B-7)}\xspace}
\newcommand\cone{\text{(C-1)}\xspace}
\newcommand\ctwo{\text{(C-2)}\xspace}
\newcommand\cthree{\text{(C-3)}\xspace}
\newcommand\cfour{\text{(C-4)}\xspace}
\newcommand\cfive{\text{(C-5)}\xspace}
\newcommand\csix{\text{(C-6)}\xspace}
\newcommand\cseven{\text{(C-7)}\xspace}






% Three sets for the primitives

\newcommand\functionset{\mathcal{F}(n)}
\newcommand\permutationset{\mathcal{P}(n)}
\newcommand\blockcipherset{\mathcal{BC}(n,2n)}

\newcommand\dis{\mathcal{D}}

\newcommand\bad{\textsf{Bad}}




\newcommand\Lvector{\ensuremath{\overrightarrow{L}}}
\newcommand\Rvector{\ensuremath{\overrightarrow{R}}}
\newcommand\Vvector{\ensuremath{\overrightarrow{V}}}
\newcommand\Wvector{\ensuremath{\overrightarrow{W}}}
\newcommand\Xvector{\ensuremath{\overrightarrow{X}}}
\newcommand\Yvector{\ensuremath{\overrightarrow{Y}}}
\newcommand\Zvector{\ensuremath{\overrightarrow{Z}}}
\newcommand\Svector{\ensuremath{\overrightarrow{S}}}
\newcommand\Tvector{\ensuremath{\overrightarrow{T}}}
\newcommand\rvector{\ensuremath{\overrightarrow{r}}}
\newcommand\vvector{\ensuremath{\overrightarrow{v}}}
\newcommand\wvector{\ensuremath{\overrightarrow{w}}}
\newcommand\xvector{\ensuremath{\overrightarrow{x}}}
\newcommand\yvector{\ensuremath{\overrightarrow{y}}}
\newcommand\zvector{\ensuremath{\overrightarrow{z}}}
\newcommand\svector{\ensuremath{\overrightarrow{s}}}



%\newcommand\Uvectorbig{\ensuremath{\overrightarrow{U}}}
%\newcommand\Vvectorbig{\ensuremath{\overrightarrow{V}}}
%\newcommand\Wvectorbig{\ensuremath{\overrightarrow{W}}}
%\newcommand\uvector{\ensuremath{\overrightarrow{u}}}
%\newcommand\vvector{\ensuremath{\overrightarrow{v}}}
%\newcommand\wvector{\ensuremath{\overrightarrow{w}}}

\newcommand\permvector{\ensuremath{\overrightarrow{\pi}}}



\newcommand\constraintset{\ensuremath{\mathcal{S}}}




% Helper functions

%\newcommand\xival{x1val}
%\newcommand\xiival{x2val}
%\newcommand\xiiival{x3val}
%\newcommand\xivval{x4val}
%\newcommand\xvval{x5val}
%\newcommand\xvival{x6val}
%\newcommand\Xval{Xval}
%\newcommand\Yval{Yval}
%\newcommand\Zval{Zval}
%\newcommand\Aval{Aval}
%\newcommand\xival{\textsf{x}_1\textsf{val}}
%%%%\newcommand\xival{\textsf{x1val}}
%%%%\newcommand\xiival{\textsf{x2val}}
%%%%\newcommand\yiival{\textsf{y2val}}
%%%%\newcommand\xiiival{\textsf{x3val}}
%%%%\newcommand\yiiival{\textsf{y3val}}
%%%%\newcommand\xivval{\textsf{x4val}}
%%%%\newcommand\xvval{\textsf{x5val}}
%%%%\newcommand\xvival{\textsf{x6val}}
%%%%\newcommand\Xval{\textsf{Xval}}
%%%%\newcommand\Yval{\textsf{Yval}}
%%%%\newcommand\Zval{\textsf{Zval}}
%%%%\newcommand\Aval{\textsf{Aval}}



% Sets of intermediate values

\newcommand\dom{\mathcal{X}}
\newcommand\rng{\mathcal{Y}}

\newcommand\extdomi{Ext\mathcal{X}_1}
\newcommand\extrngi{Ext\mathcal{Y}_1}

\newcommand\extdomii{Ext\mathcal{X}_2}
\newcommand\extrngii{Ext\mathcal{Y}_2}

\newcommand\extdomiii{Ext\mathcal{X}_3}
\newcommand\extrngiii{Ext\mathcal{Y}_3}

\newcommand\extdomiv{Ext\mathcal{X}_4}
\newcommand\extrngiv{Ext\mathcal{Y}_4}

\newcommand\extdomv{Ext\mathcal{X}_5}
\newcommand\extrngv{Ext\mathcal{Y}_5}

\newcommand\extdomvi{Ext\mathcal{X}_6}
\newcommand\extrngvi{Ext\mathcal{Y}_6}

\newcommand\extdomx{Ext\mathcal{X}_i}
\newcommand\extrngx{Ext\mathcal{Y}_i}



\newcommand\expectation{\ensuremath{\mathbb{E}}\xspace}
\newcommand\nsolution{\ensuremath{h}\xspace}
\newcommand\blocksize{\ensuremath{s}\xspace}
\newcommand\Lblocksize{\ensuremath{S}\xspace}
\newcommand\lambdavector{\ensuremath{\overrightarrow{\lambda}}\xspace}



\renewcommand\theenumi{\roman{enumi}}
\renewcommand\labelenumi{(\theenumi)}


%
% Insert the name of "your journal" with
% \journalname{myjournal}
%
\begin{document}


\title{Related-Key Security Analysis of Generalized Feistel Networks%\thanks{Grants or other notes
%about the article that should go on the front page should be
%placed here. General acknowledgments should be placed at the end of the article.}
}


\titlerunning{Related-Key Security Analysis of Generalized Feistel Networks}        % if too long for running head



\author{... \and Chun Guo}




\institute{Chun Guo \at
				Key Laboratory of Cryptologic Technology and Information Security, Ministry of Education, Shandong University     \\     
              School of Cyber Science and Technology, Shandong University      \\
              \email{chun.guo@sdu.edu.cn} 
%           \and
%           Yaobin Shen \at
%           	  \email{yb\_shen@sjtu.edu.cn}
}

\date{Received: date / Accepted: date}
% The correct dates will be entered by the editor


\maketitle

\begin{abstract}
%	We examine the related-key security of various generalized Feistel networks, including contracting, expanding, alternating Feistel, and Naor-Reingold construction. In detail,
\begin{itemize}
	\item Assuming using related-key secure round functions, we show that alternating key assignments are sufficient for contracting, expanding, and alternating Feistel networks. We also unveil the nearly optimal round complexities for the schemes that are sufficient for related-key CCA security.
	\item Assuming using related-key secure AXU hashing (introduce by Wang et al., FSE 2016) and round functions, we prove that the Naor-Reingold construction is related-key secure.
\end{itemize}
Our results enable constructing efficient variable input length related-key secure SPRPs for various applications such as non malleable codes and temper resilience.

\keywords{Blockcipher \and related-key \and Feistel networks}
% \PACS{PACS code1 \and PACS code2 \and more}
 \subclass{94A60 \and 68P25}
\end{abstract}



%

We generalize the provable RKA security of balanced Feistel networks to various
generalized Feistel networks.
These provide new theory tools.
Multi-line GFNs provide domain extender for RKA secure blockciphers, while
condensing and expanding GFNs provide more convenient ways to use the RKA
secure PRFs built upon complexity assumptions (note that such PRFs, e.g., in,
are typically of \abnormal" domain and range rather than f0; 1g).




\medskip\noindent{\bf Further Related Work.}
Various MACs ...



\floatstyle{boxed} 
\restylefloat{figure}


%
%

\section{Preliminaries}
\label{section:preliminary}


\subsection{General Notations and Definitions}

For a finite set $\mathcal{X}$, $X \xleftarrow{\$} \mathcal{X}$ denotes selecting an element from $\mathcal{X}$ uniformly at random and $|\mathcal{X}|$ denotes its cardinality.
In all the following, we fix an integer $n\geq 1$.
% and denote $N=2^n$. 
Further denote by  the set of all
functions of domain $\{0,1\}^*$ and range $\{0,1\}^{2n}$,
by $\permutationset$ the set of all
permutations on $\{0,1\}^n$, and by
$\blockcipherset$ the set of all blockciphers with $n$-bit
block-size and $n$-bit keys. Finally, for $U,X\in\{0,1\}^n$, $U\|X$ or simply $UX$ denotes their concatenation.




\section{$m=kn$}
\subsection{5-rounds}
CCA Security : The 5-Round Contracting Feistel Construction\\
We proved 5-round contracting feistel construction which can achieve CCA security under related-key attacks with the simple key assignments:$[1,2,2,1,2]$.\\
\begin{equation}
\frac{\Pr_{re}(\tau)}{\Pr_{id}(\tau)}\geq 1-( \frac{2q^{2}}{2^{n}}+\frac{3q}{2^{n}} +\frac{q^{2}}{2^{3n}})
\end{equation}
Proof:
To prove Eq.(1),we fix a transcript $\tau=(\phi_{i},X_{1,i}[1,3n],X_{6,i}[1,3n]),i=1\cdots q$,then distinguish good and bad key with respect to $\tau$, and finally analyze the probability of the good key to get the advantage of this construction.\\

% a modification
%%yuwenqi
%%zhaoyuqing
First, we classify $\phi_{i}$, suppose that the quantity of $\phi^{(i)}$ is $\alpha$.\\
$\mathcal{Q}_{1}={(\phi^{(1)},X_{11}[1,3n],X_{61}[1,3n]),\dots,(\phi^{(1)},X_{1q_{1}}[1,3n],X_{6q_{1}}[1,3n])}$ which has $q_{1}$ different input totally.\\
$\mathcal{Q}_{2}={(\phi^{(2)},X_{12}[1,3n],X_{62}[1,3n]),\dots,(\phi^{(2)},X_{1q_{2}}[1,3n],X_{6q_{2}}[1,3n])}$    which has $q_{2}$ different input totally.\\
\quad \quad $\vdots$ \\
$\mathcal{Q}_{\alpha}={(\phi^{(\alpha)},X_{1\alpha}[1,3n],X_{6\alpha}[1,3n]),\dots,(\phi^{(\alpha)},X_{1q_{\alpha}}[1,3n],X_{6q_{\alpha}}[1,3n])}$ which has $q_{\alpha}$ different input totally.\\

Suppose that there are $\alpha$ $\phi^{(i)}$ which can derive $\beta$ different $\mathcal{K}_{1}$: $\mathcal{K}_{1}^{(1)}$, $\mathcal{K}_{1}^{(2)}$, $\dots$,$\mathcal{K}_{1}^{(\beta)}$$(\beta\leq \alpha)$.\\
Suppose that q Related-Key Oracle query constitute of $q_{1}^{\ast} \dots q_{\beta}^{\ast}$ inputs separately in $\beta$ different $\mathcal{K}_{1}$.\\

Bad Keys are now defined as follows.\\
Definition 1: $\mathcal{K}_{1}^{bad}$  for 5 Rounds
with respect to $\tau$, $\mathcal{K}_{1}$  is {\it bad}, if at least one of the following conditions is fulfilled\\
(B-1)there exists $i$ and $j$ such that $X_{2,i}[n+1,3n]$ =$X_{2,j}[n+1,3n],(i\neq j)$\\
(B-2)there exists $i$ and $j$ such that $X_{2,i}[n+1,3n]$ =$X_{3,j}[n+1,3n]$\\
(B-3)there exists $i$ and $j$ such that $X_{2,i}[n+1,3n]$ =$X_{5,j}[n+1,3n]$\\
(B-4)there exists $i$ and $j$ such that $X_{3,i}[n+1,3n]$ =$X_{3,j}[n+1,3n]],(i\neq j)$\\
(B-5)there exists $i$ and $j$ such that $X_{3,i}[n+1,3n]$ =$X_{5,j}[n+1,3n]$\\
Otherwise we say $\mathcal{K}_{1}$ is {\it good}.\\

Definition 2: $\mathcal{K}_{2}^{bad}$  for 5 Rounds
with respect to $\tau$, $\mathcal{K}_{2}$  is {\it bad}, if at least one of the following conditions is fulfilled \\
(B-1)there exists $i$ and $j$ such that $X_{2,i}[n+1,3n]$ =$X_{4,j}[n+1,3n]$\\
(B-2)there exists $i$ and $j$ such that $X_{4,i}[n+1,3n]$ =$X_{4,j}[n+1,3n],(i\neq j)$\\
Otherwise we say $\mathcal{K}_{2}$ is {\it good}.\\

Now we analyze the situation of $\mathcal{K}_{1}^{bad}$.\\
Firstly, we pay attention to (B-1). If $\mathcal{K}_{1,i}\neq\mathcal{K}_{1,j}$, the probability of the collision of X is $\frac{1}{2^{n}}$. If $\mathcal{K}_{1,i}=\mathcal{K}_{1,j}$ and $\phi^{(i)}\neq\phi^{(j)}$, because of the claw free, we don't need to think about the collision. Next we think about the situation of $\phi^{(i)}=\phi^{(j)}$. There are two possible situations $X_{1,i}[2n+1,3n]=X_{1,j}[2n+1,3n]$ and $X_{1,i}[2n+1,3n]\neq X_{1,j}[2n+1,3n]$. When $X_{1,i}[2n+1,3n]\neq X_{1,j}[2n+1,3n]$, there are definitely no collision.When $X_{1,i}[2n+1,3n]=X_{1,j}[2n+1,3n]$, we think about the collision of X.
If $X_{1,i}[n+1,2n]=X_{1,j}[n+1,2n]$, because of the difference of $X[1,n]$, we can know that $X_{2,i}[2n+1,3n]\neq X_{2,j}[2n+1,3n]$. If $X_{1,i}[n+1,2n]\neq X_{1,j}[n+1,2n]$, we can get the formulation
$\Pr[X_{2,i}[2n+1,3n]=X_{2,j}[2n+1,3n]]=\frac{1}{2^{n}}$.
Further, the probability of the situation of (B-1) is $\frac{1}{2^{n}}$.

Secondly, we think about (B-2),(B-3),(B-5). This is a random collision.
So we have the formulation.
$\Pr[X_{2,i}[n+1,3n]=X_{3,j}[n+1,3n]]=\frac{1}{2^{2n}}$.\\
$\Pr[X_{2,i}[n+1,3n]=X_{5,j}[n+1,3n]]=\frac{1}{2^{2n}}$.\\
$\Pr[X_{3,i}[n+1,3n]=X_{5,j}[n+1,3n]]=\frac{1}{2^{2n}}$.\\

Finally, we pay attention to (B-4).\\
When $X_{1,i}[n+1,2n]=X_{1,j}[n+1,2n]$ and $X_{1,i}[2n+1,3n]=X_{1,j}[2n+1,3n]$, because of the difference of L,we can know that $X_{2,i}[2n+1,3n]\neq X_{2,j}[2n+1,3n]$.So $X_{3,i}[n+1,3n]\neq X_{3,j}[n+1,3n]$. \\
When $X_{1,i}[n+1,2n]\neq X_{1,j}[n+1,2n]$ and $X_{1,i}[2n+1,3n]=X_{1,j}[2n+1,3n]$, then $\Pr[X_{2,i}[2n+1,3n]=X_{2,j}[2n+1,3n]]=\frac{1}{2^{n}}$, because of the difference of $X_{1,i}[n+1,2n]$,we can know that if $X_{2,i}[2n+1,3n]=X_{2,j}[2n+1,3n]$,then there is no collision of Y.\\
When $X_{1,i}[n+1,2n]=X_{1,j}[n+1,2n]$ and $X_{1,i}[2n+1,3n]\neq X_{1,j}[2n+1,3n]$, then $\Pr[X_{2,i}[2n+1,3n]=X_{2,j}[2n+1,3n]]=\frac{1}{2^{n}}$. Further,if $X_{2,i}[2n+1,3n]=X_{2,j}[2n+1,3n]$,then $\Pr[X_{3,i}[2n+1,3n]=X_{3,j}[2n+1,3n]]=\frac{1}{2^{n}}$.So $\Pr[X_{3,i}[n+1,3n]=X_{3,j}[n+1,3n]]=\frac{1}{2^{2n}}$\\
When$X_{1,i}[n+1,2n]\neq X_{1,j}[n+1,2n]$ and $X_{1,i}[2n+1,3n]\neq X_{1,j}[2n+1,3n]$, we can get the probability is $$\Pr[X_{3,i}[n+1,3n]=X_{3,j}[n+1,3n]]=\frac{1}{2^{2n}}$$
So the probability of the situation of (B-4) is $\frac{2}{2^{2n}}$.\\
For $\mathcal{K}_{1}^{bad}$,we can get the probability:\\
$\Pr[\mathcal{K}_{1}^{bad}]\leq (\frac{1}{2^{n}}+\frac{5}{2^{2n}})q^{2}$\\

For $\mathcal{K}_{2}^{bad}$,we can get the probability:\\
$\Pr[\mathcal{K}_{2}^{bad}]\leq (\frac{1}{2^{n}}+\frac{1}{2^{2n}})q^{2}$

Lowering Bounding the Probability for Good Keys\\
We now lower bound the probability for good keys.\\
$\Pr_{re}[\tau]=(1-Pr[\mathcal{K}_{1}^{bad}])(\frac{1}{2^{n}})^{3q}\times(1-\frac{1}{2^{n}})^{3q}
(1-\Pr[\mathcal{K}_{2}^{bad}])$

\begin{align*}
\frac{\Pr_{re}(\tau)}{\Pr_{id}(\tau)}&= (1-\Pr[\mathcal{K}_{1}^{bad}])(\frac{1}{2^{n}})^{3q}\times(1-\frac{1}{2^{n}})^{3q}\times
(1-\Pr[\mathcal{K}_{2}^{bad}]) / \prod_{i=0}^{\alpha}\frac{1}{(2^{3n})_{q_{i}}}\\
&\geq (1-(\frac{1}{2^{n}}+\frac{5}{2^{2n}})q^{2})(\frac{1}{2^{n}})^{3q}\times(1-\frac{1}{2^{n}})^{3q}\times
(1-(\frac{1}{2^{n}}+\frac{1}{2^{2n}})q^{2})(2^{3n}-q)^{q}\\
&\geq(1-(\frac{1}{2^{n}}+\frac{5}{2^{2n}})q^{2})(1-(\frac{1}{2^{n}}+\frac{1}{2^{2n}})q^{2})(1-\frac{3q}{2^{n}})(1-\frac{q^{2}}{2^{3n}})\\
&\geq 1-( \frac{2q^{2}}{2^{n}}+\frac{3q}{2^{n}} +\frac{q^{2}}{2^{3n}})
\end{align*}
So if $q \ll 2^{\frac{n}{2}}$,we can get this construction is CCA-security.


\subsection{6-rounds}
We proved 6-round contracting feistel construction which can achieve CCA security under related-key attacks with the simple key assignments:$[1,2,1,2,1,2]$.\\
\begin{equation}
\frac{\Pr_{re}(\tau)}{Pr_{id}(\tau)}\geq 1-( \frac{2q^{2}}{2^{n}}+\frac{3q}{2^{n}} +\frac{q^{2}}{2^{3n}})
\end{equation}
Proof:
To prove Eq.(2),we distinguish good and bad key with respect to $\tau$, and finally analyze the probability of the good key to get the advantage of this construction.\\

First, we classify $\phi_{i}$, suppose that the quantity of $\phi^{(i)}$ is $\alpha$.\\
$\mathcal{Q}_{1}={(\phi^{(1)},X_{11}[1,4n],X_{71}[1,4n]),\dots,(\phi^{(1)},X_{1q_{1}}[1,4n],X_{7q_{1}}[1,4n])}$ which has $q_{1}$ different input totally.\\
$\mathcal{Q}_{2}={(\phi^{(2)},X_{12}[1,3n],X_{72}[1,4n]),\dots,(\phi^{(2)},X_{1q_{2}}[1,4n],X_{7q_{2}}[1,4n])}$    which has $q_{2}$ different input totally.\\
\quad \quad $\vdots$ \\
$\mathcal{Q}_{\alpha}={(\phi^{(\alpha)},X_{1\alpha}[1,4n],X_{7\alpha}[1,4n]),\dots,(\phi^{(\alpha)},X_{1q_{\alpha}}[1,4n],X_{7q_{\alpha}}[1,4n])}$ which has $q_{\alpha}$ different input totally.\\

Suppose that there are $\alpha$ $\phi^{(i)}$ which can derive $\beta$ different $\mathcal{K}_{1}$: $\mathcal{K}_{1}^{(1)}$, $\mathcal{K}_{1}^{(2)}$, $\dots$,$\mathcal{K}_{1}^{(\beta)}$$(\beta\leq \alpha)$.\\
Suppose that q Related-Key Oracle query constitute of $q_{1}^{\ast} \dots q_{\beta}^{\ast}$ inputs separately in $\beta$ different $\mathcal{K}_{1}$.\\

Bad Keys are now defined as follows.\\
Definition 1: $\mathcal{K}_{1}^{bad}$  for 6 Rounds
with respect to $\tau$, $\mathcal{K}_{1}$  is {\it bad}, if at least one of the following conditions is fulfilled\\
(B-1)there exists $i$ and $j$ such that $X_{2,i}[n+1,4n]$ =$X_{2,j}[n+1,4n],(i\neq j)$\\
(B-2)there exists $i$ and $j$ such that $X_{2,i}[n+1,4n]$ =$X_{4,j}[n+1,4n]$\\
(B-3)there exists $i$ and $j$ such that $X_{2,i}[n+1,4n]$ =$X_{6,j}[n+1,4n]$\\
(B-4)there exists $i$ and $j$ such that $X_{4,i}[n+1,4n]$ =$X_{4,j}[n+1,4n],(i\neq j)$\\
(B-5)there exists $i$ and $j$ such that $X_{4,i}[n+1,4n]$ =$X_{6,j}[n+1,4n]$\\
Otherwise we say $\mathcal{K}_{1}$ is {\it good}.\\

Definition 2: $\mathcal{K}_{2}^{bad}$  for 6 Rounds
with respect to $\tau$, $\mathcal{K}_{2}$  is {\it bad}, if at least one of the following conditions is fulfilled \\
(B-1)there exists $i$ and $j$ such that $X_{5,i}[n+1,4n]$ =$X_{5,j}[n+1,4n],(i\neq j)$\\
(B-2)there exists $i$ and $j$ such that $X_{5,i}[n+1,4n]$ =$X_{3,j}[n+1,4n]$\\
(B-3)there exists $i$ and $j$ such that $X_{5,i}[n+1,4n]$ =$X_{1,j}[n+1,4n]$\\
(B-4)there exists $i$ and $j$ such that $X_{3,i}[n+1,4n]$ =$X_{3,j}[n+1,4n],(i\neq j)$\\
(B-5)there exists $i$ and $j$ such that $X_{3,i}[n+1,4n]$ =$X_{1,j}[n+1,4n]$\\
Otherwise we say $\mathcal{K}_{2}$ is {\it good}.\\

For $\mathcal{K}_{1}^{bad}$,we can get the probability:\\
$\Pr[\mathcal{K}_{1}^{bad}]\leq (\frac{5}{2^{n}})q^{2}$\\

For $\mathcal{K}_{2}^{bad}$,we can get the probability:\\
$\Pr[\mathcal{K}_{2}^{bad}]\leq (\frac{5}{2^{n}})q^{2}$\\
Lowering Bounding the Probability for Good Keys\\
We now lower bound the probability for good keys.\\
$\Pr_{re}[\tau]=(1-\Pr[\mathcal{K}_{1}^{bad}])(\frac{1}{2^{n}})^{4q}\times(1-\frac{1}{2^{n}})^{4q}
(1-\Pr[\mathcal{K}_{2}^{bad}])$

\begin{align*}
\frac{\Pr_{re}(\tau)}{\Pr_{id}(\tau)}&= (1-\Pr[\mathcal{K}_{1}^{bad}])\times(\frac{1}{2^{n}})^{4q}\times(1-\frac{1}{2^{n}})^{4q}\times
(1-\Pr[\mathcal{K}_{2}^{bad}]) / \prod_{i=0}^{\alpha}\frac{1}{(2^{4n})_{q_{i}}}\\
&\geq (1-\frac{5q^{2}}{2^{n}})\times(\frac{1}{2^{n}})^{4q}\times(1-\frac{1}{2^{n}})^{4q}\times
(1-\frac{5q^{2}}{2^{n}})(2^{4n}-q)^{q}\\
&\geq(1-\frac{10q^{2}}{2^{n}})(1-\frac{4q}{2^{n}})(1-\frac{q^{2}}{2^{4n}})\\
&\geq 1-( \frac{10q^{2}}{2^{n}}+\frac{4q}{2^{n}} +\frac{q^{2}}{2^{4n}})
\end{align*}
So if $q \ll 2^{\frac{n}{2}}$,we can get this construction is CCA-security.\\
\\
\\
\\

\section{$m=kn+r$}

\subsection{k is even}
Based on the above discussion, we expand this issue to general situation. When $\frac{m}{n}=k+r, k$ is a integer, we discuss the CCA security of k+3+1 rounds with respect to odd and even rounds separately.\\

Firstly, When k is even, we talk about the CCA security under related-key attacks with the simple key assignments:$[1,2,1,2,\dots,1,2]$.\\

\begin{equation}
\frac{\Pr_{re}(\tau)}{Pr_{id}(\tau)}\geq 1-( \frac{q^{2}}{2^{(k+2)n}}+\frac{(k+2)q}{2^{n}} +\frac{(k^{2}+10k+16)q^{2}}{2^{n+2}})
\end{equation}

Proof:
To prove Eq.(3),we distinguish good and bad key with respect to $\tau$, and finally analyze the probability of the good key to get the advantage of this construction.\\
\\
First, we classify $\phi_{i}$, suppose that the quantity of $\phi^{(i)}$ is $\alpha$.\\
$\mathcal{Q}_{1}={(\phi^{(1)},X_{11}[1,(k+1)n],X_{(k+4)1}[1,(k+1)n]),\dots,(\phi^{(1)},X_{1q_{1}}[1,(k+1)n],X_{(k+4)q_{1}}[1,(k+1)n])}$ which has $q_{1}$ different input totally.\\
$\mathcal{Q}_{2}={(\phi^{(2)},X_{12}[1,(k+1)n],X_{(k+4)2}[1,(k+1)n]),\dots,(\phi^{(2)},X_{1q_{2}}[1,(k+1)n],X_{(k+4)q_{2}}[1,(k+1)n])}$    which has $q_{2}$ different input totally.\\
\quad \quad $\vdots$ \\
$\mathcal{Q}_{\alpha}={(\phi^{(\alpha)},X_{1\alpha}[1,(k+1)n],X_{(k+4)\alpha}[1,(k+1)n]),\dots,(\phi^{(\alpha)},X_{1q_{\alpha}}[1,(k+1)n],X_{(k+4)q_{\alpha}}[1,(k+1)n])}$ which has $q_{\alpha}$ different input totally.\\

Suppose that there are $\alpha$ different $\phi^{(i)}$ which can derive $\beta$ different $\mathcal{K}_{1}$: $\mathcal{K}_{1}^{(1)}$, $\mathcal{K}_{1}^{(2)}$, $\dots$,$\mathcal{K}_{1}^{(\beta)}$$(\beta\leq \alpha)$.\\\\
Suppose that q Related-Key Oracle query constitute of $q_{1}^{\ast} \dots q_{\beta}^{\ast}$ inputs separately in $\beta$ different $\mathcal{K}_{1}$.\\
\\

Bad Keys are now defined as follows.\\
Definition 1: $\mathcal{K}_{1}^{bad}$  for k+3+1 Rounds
with respect to $\tau$, $\mathcal{K}_{1}$  is {\it bad}, if at least one of the following conditions is fulfilled\\
(B-1)there exists $i$ and $j$ such that $X_{2,i}[n+1,(k+1)n+r]$=$X_{2,j}[n+1,(k+1)n+r],(i\neq j)$\\
(B-2)there exists $i$ and $j$ such that $X_{2,i}[n+1,(k+1)n+r]$=$X_{4,j}[n+1,(k+1)n+r]$\\
\dots\\
(B-$\frac{k+4}{2}$)there exists $i$ and $j$ such that $X_{2,i}[n+1,(k+1)n+r]$=$X_{k+3,j}[n+1,(k+1)n+r]$\\
(B-$\frac{k+4}{2}$+1)there exists $i$ and $j$ such that $X_{4,i}[n+1,(k+1)n+r]$=$X_{4,j}[n+1,(k+1)n+r],(i\neq j)$\\
(B-$\frac{k+4}{2}$+2)there exists $i$ and $j$ such that $X_{4,i}[n+1,(k+1)n+r]$=$X_{6,j}[n+1,(k+1)n+r]$\\
\dots \\
(B-$k+3$)there exists $i$ and $j$ such that $X_{4,i}[n+1,(k+1)n+r]$=$X_{k+4,j}[n+1,(k+1)n+r]$\\
\dots \\
(B-$\frac{k^{2}+10k+8}{8}$)there exists $i$ and $j$ such that $X_{k+2,i}[n+1,(k+1)n+r]$=$X_{k+2,j}[n+1,(k+1)n+r]$\\
(B-$\frac{k^{2}+10k+16}{8}$)there exists $i$ and $j$ such that $X_{k+2,i}[n+1,(k+1)n+r]$=$X_{k+4,j}[n+1,(k+1)n+r]$\\
Otherwise we say $\mathcal{K}_{1}$ is {\it good}.\\

For $\mathcal{K}_{1}^{bad}$,we can get the probability:\\
$\Pr[\mathcal{K}_{1}^{bad}]\leq (\frac{k^{2}+10k+16}{8\times2^{n}})q^{2}$\\

For $\mathcal{K}_{2}^{bad}$, we have similar results:\\
$\Pr[\mathcal{K}_{2}^{bad}]\leq (\frac{k^{2}+10k+16}{8\times2^{n}})q^{2}$\\

Lowering Bounding the Probability for Good Keys\\
We now lower bound the probability for good keys.\\
$\Pr_{re}[\tau]=(1-\Pr[\mathcal{K}_{1}^{bad}])(\frac{1}{2^{(k+1)n+r}})^{q}\times(1-\frac{1}{2^{n}})^{(k+2)q}
(1-\Pr[\mathcal{K}_{2}^{bad}])$

\begin{align*}
\frac{\Pr_{re}(\tau)}{\Pr_{id}(\tau)}&= (1-\Pr[\mathcal{K}_{1}^{bad}])\times(\frac{1}{2^{(k+1)n+r}})^{q}\times(1-\frac{1}{2^{n}})^{(k+2)q}\times
(1-\Pr[\mathcal{K}_{2}^{bad}]) / \prod_{i=0}^{\alpha}\frac{1}{(2^{(k+1)n+r})_{q_{i}}}\\
&\geq (1-\frac{k^{2}+10k+16}{8\times2^{n}}q^{2})\times(\frac{1}{2^{(k+1)n+r}})^{q}\times(1-\frac{1}{2^{n}})^{(k+2)q}\times
(1-\frac{k^{2}+10k+16}{8\times2^{n}}q^{2})(2^{(k+1)n+r}-q)^{q}\\
&\geq 1-( \frac{q^{2}}{2^{(k+2)n}}+\frac{(k+2)q}{2^{n}} +\frac{(k^{2}+10k+16)q^{2}}{2^{n+2}})
\end{align*}
So if $q \ll 2^{\frac{n}{2}}$,we can get this construction is CCA-security.\\


\subsection{k is odd}

Secondly, When k is odd, we talk about the CCA security under related-key attacks with the simple key assignments:$[1,2,2,1,2,\dots,1,2]$.\\

\begin{equation}
\frac{\Pr_{re}(\tau)}{Pr_{id}(\tau)}\geq 1-( \frac{q^{2}}{2^{(k+1)n+r}}+\frac{(k+2)q}{2^{n}} +\frac{(k^{2}+12k+27)q^{2}}{2^{n+2}})
\end{equation}

Proof:
To prove Eq.(4),we distinguish good and bad key with respect to $\tau$, and finally analyze the probability of the good key to get the advantage of this construction.\\
\\
First, we classify $\phi_{i}$, suppose that the quantity of $\phi^{(i)}$ is $\alpha$.\\
$\mathcal{Q}_{1}={(\phi^{(1)},X_{11}[1,(k+1)n],X_{(k+4)1}[1,(k+1)n]),\dots,(\phi^{(1)},X_{1q_{1}}[1,(k+1)n],X_{(k+4)q_{1}}[1,(k+1)n])}$ which has $q_{1}$ different input totally.\\
$\mathcal{Q}_{2}={(\phi^{(2)},X_{12}[1,(k+1)n],X_{(k+4)2}[1,(k+1)n]),\dots,(\phi^{(2)},X_{1q_{2}}[1,(k+1)n],X_{(k+4)q_{2}}[1,(k+1)n])}$    which has $q_{2}$ different input totally.\\
\quad \quad $\vdots$ \\
$\mathcal{Q}_{\alpha}={(\phi^{(\alpha)},X_{1\alpha}[1,(k+1)n],X_{(k+4)\alpha}[1,(k+1)n]),\dots,(\phi^{(\alpha)},X_{1q_{\alpha}}[1,(k+1)n],X_{(k+4)q_{\alpha}}[1,(k+1)n])}$ which has $q_{\alpha}$ different input totally.\\

Suppose that there are $\alpha$ different $\phi^{(i)}$ which can derive $\beta$ different $\mathcal{K}_{1}$: $\mathcal{K}_{1}^{(1)}$, $\mathcal{K}_{1}^{(2)}$, $\dots$,$\mathcal{K}_{1}^{(\beta)}$$(\beta\leq \alpha)$.\\\\
Suppose that q Related-Key Oracle query constitute of $q_{1}^{\ast} \dots q_{\beta}^{\ast}$ inputs separately in $\beta$ different $\mathcal{K}_{1}$.\\


Bad Keys are now defined as follows.\\
Definition 1: $\mathcal{K}_{1}^{bad}$  for k+3+1 Rounds
with respect to $\tau$, $\mathcal{K}_{1}$  is {\it bad}, if at least one of the following conditions is fulfilled\\
(B-1)there exists $i$ and $j$ such that $X_{2,i}[n+1,(k+1)n+r]$=$X_{2,j}[n+1,(k+1)n+r],(i\neq j)$\\
(B-2)there exists $i$ and $j$ such that $X_{2,i}[n+1,(k+1)n+r]$=$X_{3,j}[n+1,(k+1)n+r]$\\
\dots\\
(B-$\frac{k+5}{2}$)there exists $i$ and $j$ such that $X_{2,i}[n+1,(k+1)n+r]$=$X_{k+4,j}[n+1,(k+1)n+r]$\\
(B-$\frac{k+5}{2}$+1)there exists $i$ and $j$ such that $X_{3,i}[n+1,(k+1)n+r]$=$X_{3,j}[n+1,(k+1)n+r],(i\neq j)$\\
(B-$\frac{k+5}{2}$+2)there exists $i$ and $j$ such that $X_{3,i}[n+1,(k+1)n+r]$=$X_{5,j}[n+1,(k+1)n+r]$\\
\dots \\
(B-$k+4$)there exists $i$ and $j$ such that $X_{3,i}[n+1,(k+1)n+r]$=$X_{k+4,j}[n+1,(k+1)n+r]$\\
\dots \\
(B-$\frac{k^{2}+12k+19}{8}$)there exists $i$ and $j$ such that $X_{k+2,i}[n+1,(k+1)n+r]$=$X_{k+2,j}[n+1,(k+1)n+r]$\\
(B-$\frac{k^{2}+12k+27}{8}$)there exists $i$ and $j$ such that $X_{k+2,i}[n+1,(k+1)n+r]$=$X_{k+4,j}[n+1,(k+1)n+r]$\\

Otherwise we say $\mathcal{K}_{1}$ is {\it good}.\\
For $\mathcal{K}_{1}^{bad}$,we can get the probability:\\
$\Pr[\mathcal{K}_{1}^{bad}]\leq (\frac{k^{2}+12k+27}{8\times2^{n}})q^{2}$\\

Bad Keys are now defined as follows.\\
Definition 2: $\mathcal{K}_{2}^{bad}$  for k+3+1 Rounds
with respect to $\tau$, $\mathcal{K}_{2}$  is {\it bad}, if at least one of the following conditions is fulfilled\\
(B-1)there exists $i$ and $j$ such that $X_{k+3,i}[n+1,(k+1)n+r]$=$X_{k+3,j}[n+1,(k+1)n+r],(i\neq j)$\\
(B-2)there exists $i$ and $j$ such that $X_{k+3,i}[n+1,(k+1)n+r]$=$X_{k+1,j}[n+1,(k+1)n+r]$\\
\dots\\
(B-$\frac{k+3}{2}$)there exists $i$ and $j$ such that $X_{k+3,i}[n+1,(k+1)n+r]$=$X_{1,j}[n+1,(k+1)n+r]$\\
(B-$\frac{k+3}{2}$+1)there exists $i$ and $j$ such that $X_{k+1,i}[n+1,(k+1)n+r]$=$X_{k+1,j}[n+1,(k+1)n+r],(i\neq j)$\\
(B-$\frac{k+3}{2}$+2)there exists $i$ and $j$ such that $X_{k+1,i}[n+1,(k+1)n+r]$=$X_{k-1,j}[n+1,(k+1)n+r]$\\
\dots \\
(B-$k+2$)there exists $i$ and $j$ such that $X_{k+1,i}[n+1,(k+1)n+r]$=$X_{1,j}[n+1,(k+1)n+r]$\\
\dots \\
(B-$\frac{k^{2}+8k-1}{8}$)there exists $i$ and $j$ such that $X_{4,i}[n+1,(k+1)n+r]$=$X_{4,j}[n+1,(k+1)n+r]$\\
(B-$\frac{k^{2}+8k+7}{8}$)there exists $i$ and $j$ such that $X_{4,i}[n+1,(k+1)n+r]$=$X_{1,j}[n+1,(k+1)n+r]$\\
(B-$\frac{k^{2}+8k+7}{8}$+1)there exists $i$ and $j$ such that $X_{3,i}[n+1,(k+1)n+r]$=$X_{2,j}[n+1,(k+1)n+r]$\\
(B-$\frac{k^{2}+8k+7}{8}$+2)there exists $i$ and $j$ such that $X_{3,i}[n+1,(k+1)n+r]$=$X_{3,j}[n+1,(k+1)n+r],(i\neq j)$\\
\dots\\
(B-$\frac{k^{2}+12k+27}{8}$)there exists $i$ and $j$ such that $X_{3,i}[n+1,(k+1)n+r]$=$X_{k+4,j}[n+1,(k+1)n+r]$\\
Otherwise we say $\mathcal{K}_{2}$ is {\it good}.\\
For $\mathcal{K}_{2}^{bad}$,we can get the probability:\\
$\Pr[\mathcal{K}_{2}^{bad}]\leq (\frac{k^{2}+12k+27}{8\times2^{n}})q^{2}$\\

Lowering Bounding the Probability for Good Keys\\
We now lower bound the probability for good keys.\\
$\Pr_{re}[\tau]=(1-\Pr[\mathcal{K}_{1}^{bad}])(\frac{1}{2^{(k+1)n+r}})^{q}\times(1-\frac{1}{2^{n}})^{(k+2)q}
(1-\Pr[\mathcal{K}_{2}^{bad}])$

\begin{align*}
\frac{\Pr_{re}(\tau)}{\Pr_{id}(\tau)}&= (1-\Pr[\mathcal{K}_{1}^{bad}])\times(\frac{1}{2^{(k+1)n+r}})^{q}\times(1-\frac{1}{2^{n}})^{(k+2)q}\times
(1-\Pr[\mathcal{K}_{2}^{bad}]) / \prod_{i=0}^{\alpha}\frac{1}{(2^{(k+1)n+r})_{q_{i}}}\\
&\geq (1-\frac{k^{2}+12k+27}{8\times2^{n}}q^{2})\times(\frac{1}{2^{(k+1)n+r}})^{q}\times(1-\frac{1}{2^{n}})^{(k+2)q}\times
(1-\frac{k^{2}+12k+27}{8\times2^{n}}q^{2})(2^{(k+1)n+r}-q)^{q}\\
&\geq 1-( \frac{q^{2}}{2^{(k+1)n+r}}+\frac{(k+2)q}{2^{n}} +\frac{(k^{2}+12k+27)q^{2}}{2^{n+2}})
\end{align*}
So if $q \ll 2^{\frac{n}{2}}$,we can get this construction is CCA-security.\\






See the following:
$$\bkks.$$



\section*{Acknowledgements}

\bibliography{crypto/reference-set,crypto/abbrev1,crypto/crypto}

\end{document}
